\documentclass[
	% -- opções da classe memoir --
	11pt,				% tamanho da fonte
	openright,			% capítulos começam em pág ímpar (insere página vazia caso preciso)
	twoside,			% para impressão em recto e verso. Oposto a oneside
	a4paper,			% tamanho do papel. 
	% -- opções da classe abntex2 --
	%chapter=TITLE,		% títulos de capítulos convertidos em letras maiúsculas
	%section=TITLE,		% títulos de seções convertidos em letras maiúsculas
	%subsection=TITLE,	% títulos de subseções convertidos em letras maiúsculas
	%subsubsection=TITLE,% títulos de subsubseções convertidos em letras maiúsculas
	% -- opções do pacote babel --
	english,			% idioma adicional para hifenização
	brazil				% o último idioma é o principal do documento
	]{abntex2}

% Pacotes básicos 
\usepackage{lmodern}		% Usa a fonte Latin Modern
\usepackage[T1]{fontenc}	% Selecao de codigos de fonte.
\usepackage{indentfirst}	% Indenta o primeiro parágrafo de cada seção.
\usepackage{color}			% Controle das cores
\usepackage{graphicx}		% Inclusão de gráficos
\usepackage{microtype} 		% para melhorias de justificação
\usepackage{lipsum}			% para geração de dummy text
\usepackage[brazilian,hyperpageref]{backref} % Paginas com as citações na bibl
\usepackage[alf]{abntex2cite} % Citações padrão ABNT

% Configurações de Pacotes
%% Configurações do pacote backref ( Usado sem a opção hyperpageref de backref )
\renewcommand{\backrefpagesname}{Citado na(s) página(s):~}
%% Texto padrão antes do número das páginas
\renewcommand{\backref}{}
%% Define os textos da citação
\renewcommand*{\backrefalt}[4]{
	\ifcase #1 
		Nenhuma citação no texto.
	\or
		Citado na página #2.
	\else
		Citado #1 vezes nas páginas #2.
	\fi}%

% Formatação da capa no padrão da USP
\renewcommand{\imprimircapa}{
\begin{capa}
    \center
        \ABNTEXchapterfont\Large UNIVERSIDADE DE SÃO PAULO \\ ESCOLA DE ENGENHARIA DE SÃO CARLOS \\ INSTITUTO DE CIÊNCIAS MATEMÁTICAS E DE COMPUTAÇÃO
        \vspace*{1cm}
        \vfill
        \begin{center}
        \ABNTEXchapterfont\bfseries\LARGE\imprimirtitulo
        \end{center}
        \vspace*{1cm}
        \Large{v.1}
        \vfill
        \large\imprimirlocal
        \\
        \large\imprimirdata
        \vspace*{1cm}
\end{capa}
}

% Formatação da folha de rosto no padrão da USP
\makeatletter
\renewcommand{\folhaderostocontent}{
\begin{center}
    {\ABNTEXchapterfont\bfseries\Large\imprimirautor}
    \vspace*{\fill}\vspace*{\fill}
    \begin{center}
    \ABNTEXchapterfont\bfseries\Large\imprimirtitulo
    \end{center}
    \vspace*{\fill}
    \abntex@ifnotempty{\imprimirpreambulo}{
    \hspace{.45\textwidth}
    \begin{minipage}{.5\textwidth}
    \SingleSpacing
    \imprimirpreambulo
    \vspace*{2cm}
    \imprimirorientadorRotulo~\imprimirorientador
    \end{minipage}
    \vspace*{\fill}
    }
    \vspace*{\fill}
    {\large\imprimirlocal}
    \par
    {\large\imprimirdata}
    \vspace*{1cm}
\end{center}
}
\makeatother

% Configurações de aparência do PDF final
%% alterando o aspecto da cor azul
\definecolor{blue}{RGB}{41,5,195}

%% informações do PDF
\makeatletter
\hypersetup{
     	%pagebackref=true,
		pdftitle={\@title}, 
		pdfauthor={\@author},
    	pdfsubject={\imprimirpreambulo},
	    pdfcreator={LaTeX with abnTeX2},
		pdfkeywords={abnt}{latex}{abntex}{abntex2}{trabalho acadêmico}, 
		colorlinks=true,       		% false: boxed links; true: colored links
    	linkcolor=blue,          	% color of internal links
    	citecolor=blue,        		% color of links to bibliography
    	filecolor=magenta,      		% color of file links
		urlcolor=blue,
		bookmarksdepth=4
}
\makeatother

% Posiciona figuras e tabelas no topo da página quando adicionadas sozinhas
% em um página em branco. Ver https://github.com/abntex/abntex2/issues/170
\makeatletter
\setlength{\@fptop}{5pt} % Set distance from top of page to first float
\makeatother

% Espaçamentos entre linhas e parágrafos 
% O tamanho do parágrafo é dado por:
\setlength{\parindent}{1.3cm}

% Controle do espaçamento entre um parágrafo e outro:
\setlength{\parskip}{0.2cm}  % tente também \onelineskip

% Compila o índice
\makeindex

% Informações de dados para a capa e folha de rosto
\titulo{Título do TCC}
\autor{Bruna Yukari Fujii Yoshida}
\local{São Carlos}
\data{2020}
\orientador{Valdir Grassi Junior}
\instituicao{Universidade do São Paulo - USP}
\preambulo{\Large{Trabalho de Conclusão de Curso apresentado à Escola de Engenharia de São Carlos, da Universidade de São Paulo \newline Curso de Engenharia de Computação}}