% ------------------------------------------------------------------------
% ------------------------------------------------------------------------
% ICMC: Modelo de Trabalho Acadêmico (tese de doutorado, dissertação de
% mestrado e trabalhos monográficos em geral) em conformidade com 
% ABNT NBR 14724:2011: Informação e documentação - Trabalhos acadêmicos -
% Apresentação
% ------------------------------------------------------------------------
% ------------------------------------------------------------------------

% Opções: 
%   Qualificação          = qualificacao 
%   Curso                 = doutorado/mestrado
%   Situação do trabalho  = pre-defesa/pos-defesa (exceto para qualificação)
%   Versão para impressão = impressao
\documentclass[qualificacao, pre-defesa]{packages/icmc}

% ---------------------------------------------------------------------------
% Pacotes Opcionais
% ---------------------------------------------------------------------------
\usepackage{rotating}           % Usado para rotacionar o texto
\usepackage[all,knot,arc,import,poly]{xy}   % Pacote para desenhos gráficos
% Este pacote pode conflitar com outros pacotes gráficos como o ``pictex''
% Então é necessário usar apenas um dos pacotes conflitantes
\newcommand{\VerbL}{0.52\textwidth}
\newcommand{\LatL}{0.42\textwidth}
% ---------------------------------------------------------------------------


% ---
% Informações de dados para CAPA e FOLHA DE ROSTO conforme padrao da eesc!
% ---
% Tanto na capa quanto nas folhas de rosto apenas a primeira letra da primeira palavra (ou nomes próprios) devem estar em letra maiúscula, todas as demais devem ser em letra minúscula.
\tituloPT{Modelo de teses e dissertações em LaTeX}
\tituloEN{Model of theses and dissertations in LaTeX}
\autor[Antonelli, H. L.]{Humberto Lidio Antonelli}
\genero{F} % Gênero do autor (M = Masculino / F = Feminino)
\orientador[Orientador]{Prof. Dr.}{Valdir Grassi Junior}
%\coorientador{Prof. Dr.}{Fulano de Tal}
\curso{ENGCOMP}
\data{12}{10}{2020} % Data do depósito
\idioma{PT} % Idioma principal do documento (PT = português / EN = inglês)
% ---



% DEDICATÓRIA / AGRADECIMENTO - opcionais
\textodedicatoria*{tex/pre-textual/dedicatoria}
\textoagradecimentos*{tex/pre-textual/agradecimentos}


% ---
% SUMARIO - OBRIGATORIO
% ---

% ----------------------------------------------------------
% ELEMENTOS PRÉ-TEXTUAIS OPCIONAIS
% ----------------------------------------------------------


% Inclui a lista de figuras
\incluilistadefiguras

% Inclui a lista de tabelas
\incluilistadetabelas

% Inclui a lista de quadros
\incluilistadequadros

% Inclui a lista de algoritmos
\incluilistadealgoritmos

% Inclui a lista de códigos
\incluilistadecodigos

% Inclui a lista de siglas e abreviaturas
\incluilistadesiglas

% Inclui a lista de símbolos
\incluilistadesimbolos

% ---
% RESUMOS
% ---

% Resumo, em português, com no máximo 1500 caracteres (com espaço)
% e seis palavras-chaves para o assunto abordado.
\textoresumo[brazil]{
    Este trabalho é um breve modelo  para a escrita de monografias de qualificação, dissertações e teses utilizando o ambiente \LaTeX, de acordo com as normas exigidas pelo Instituto de Ciências Matemáticas e de Computação (ICMC), da Universidade de São Paulo (USP). Para a confecção deste modelo foi utilizado a última versão (1.9.6) do pacote de classes \textit{abnTeX2} que segue as normas da Associação Brasileira de Normas Técnicas. A elaboração de uma monografia, dissertação ou tese pode ser feita sobrescrevendo o conteúdo deste modelo.
    }{Modelo, Monografia de qualificação, Dissertação, Tese, Latex}


% Abstract, em inglês, com no máximo 1500 caracteres (com espaço)
% e seis keywords para o assunto abordado.
\textoresumo[english]{
    This paper is a brief model for writing qualification monographs, dissertations and thesis using \LaTeX environment, in accordance with the standards required by the Institute of Mathematics and Computer Sciences (ICMC), University of São Paulo (USP). For making this model, the latest version (1.9.6) \textit{abnTeX2} classes package was used. This package follow the rules of the Brazilian Association of Technical Standards. A drafting a monograph, dissertation or thesis can be done by overwriting the contents of this model.
    }{Template, Qualification monograph, Dissertation, Thesis, Latex}


% ----
% Início do documento
% ----
\begin{document}
% ----------------------------------------------------------
% ELEMENTOS TEXTUAIS
% ----------------------------------------------------------
\textual

% O trabalho deve conter:
%(a) Apresentação do problema central do trabalho de formatura;
%(b) Formulação no formato de um problema de engenharia;
%(c) Definição de escopo;
%(d) Revisão de literatura;
%(e) Desenvolvimento metodológico (inclusive com a parte experimental quando pertinente);

\chapter{Introdução}
\label{chapter:introducao}
% Comando simples para exibir comandos Latex no texto
\newcommand{\comando}[1]{\textbf{$\backslash$#1}}

Introducao

% Corpo do trabalho (demais capítulos)
\chapter{Revisao de Literatura}
\label{chapter:revisao-literatura}
\input{tex/revisao-literatura}

\chapter{Devenvolvimento metodológico}
\label{chapter:desenvolvimento-metodologico}
\input{tex/desenvolvimento-metodologico}

\chapter{Conclusões}
\label{chapter:conclusoes}
\input{tex/conclusoes}

% ---
% Finaliza a parte no bookmark do PDF, para que se inicie o bookmark na raiz
% ---
\bookmarksetup{startatroot}%
% ---

% ----------------------------------------------------------
% ELEMENTOS PÓS-TEXTUAIS
% ----------------------------------------------------------
\postextual

% ----------------------------------------------------------
% Anexos - Quando couber
% ----------------------------------------------------------

% ---
% Inicia os anexos
% ---
\begin{anexosenv}

    \chapter{Páginas interessantes na Internet}
    \label{chapter:paginas-interessantes}
    \input{tex/annex/paginas-interessantes}

\end{anexosenv}
% ---

% ----------------------------------------------------------
% Referências bibliográficas
% ----------------------------------------------------------
\bibliography{references}

% ----------------------------------------------------------
% Bibliografia Consultada - OPCIONAL
% ----------------------------------------------------------

% ----------------------------------------------------------
% Apêndices - OPCIONAL
% ----------------------------------------------------------

% ---
% Inicia os apêndices
% ---
\begin{apendicesenv}

    \chapter{Documento básico usando a classe \textit{icmc}}
    \label{chapter:documento-basico}
    \input{tex/appendix/documento-basico}

    \chapter{Configuração do programa JabRef}
    \label{chapter:configuracao-jabref}
    \input{tex/appendix/configuracao-jabref}

\end{apendicesenv}
% ---

% ---------------------------------------------------------------------
% GLOSSÁRIO -OPCIONAL
% ---------------------------------------------------------------------

% Arquivo que contém as definições que vão aparecer no glossário
\input{tex/glossario}
% Comando para incluir todas as definições do arquivo glossario.tex
\glsaddall
% Impressão do glossário
\printglossaries


\end{document}