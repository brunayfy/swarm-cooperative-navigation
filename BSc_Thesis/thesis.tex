% ------------------------------------------------------------------------
% ------------------------------------------------------------------------
% ICMC: Modelo de Trabalho Acadêmico (tese de doutorado, dissertação de
% mestrado e trabalhos monográficos em geral) em conformidade com 
% ABNT NBR 14724:2011: Informação e documentação - Trabalhos acadêmicos -
% Apresentação
% ------------------------------------------------------------------------
% ------------------------------------------------------------------------

% Opções: 
%   Qualificação          = qualificacao 
%   Curso                 = doutorado/mestrado
%   Situação do trabalho  = pre-defesa/pos-defesa (exceto para qualificação)
%   Versão para impressão = impressao
\documentclass[qualificacao, pre-defesa]{packages/icmc}

% ---------------------------------------------------------------------------
% Pacotes Opcionais
% ---------------------------------------------------------------------------
\usepackage{rotating}           % Usado para rotacionar o texto
\usepackage[all,knot,arc,import,poly]{xy}   % Pacote para desenhos gráficos
% Este pacote pode conflitar com outros pacotes gráficos como o ``pictex''
% Então é necessário usar apenas um dos pacotes conflitantes
\newcommand{\VerbL}{0.52\textwidth}
\newcommand{\LatL}{0.42\textwidth}
% ---------------------------------------------------------------------------


% ---
% Informações de dados para CAPA e FOLHA DE ROSTO conforme padrao da eesc!
% ---
% Tanto na capa quanto nas folhas de rosto apenas a primeira letra da primeira palavra (ou nomes próprios) devem estar em letra maiúscula, todas as demais devem ser em letra minúscula.
\tituloPT{Modelo de teses e dissertações em LaTeX}
\tituloEN{Model of theses and dissertations in LaTeX}
\autor[Antonelli, H. L.]{Humberto Lidio Antonelli}
\genero{F} % Gênero do autor (M = Masculino / F = Feminino)
\orientador[Orientador]{Prof. Dr.}{Valdir Grassi Junior}
%\coorientador{Prof. Dr.}{Fulano de Tal}
\curso{ENGCOMP}
\data{12}{10}{2020} % Data do depósito
\idioma{PT} % Idioma principal do documento (PT = português / EN = inglês)
% ---



% DEDICATÓRIA / AGRADECIMENTO - opcionais
\textodedicatoria*{tex/pre-textual/dedicatoria}
\textoagradecimentos*{tex/pre-textual/agradecimentos}


% ---
% SUMARIO - OBRIGATORIO
% ---

% ----------------------------------------------------------
% ELEMENTOS PRÉ-TEXTUAIS OPCIONAIS
% ----------------------------------------------------------


% Inclui a lista de figuras
\incluilistadefiguras

% Inclui a lista de tabelas
\incluilistadetabelas

% Inclui a lista de quadros
\incluilistadequadros

% Inclui a lista de algoritmos
\incluilistadealgoritmos

% Inclui a lista de códigos
\incluilistadecodigos

% Inclui a lista de siglas e abreviaturas
\incluilistadesiglas

% Inclui a lista de símbolos
\incluilistadesimbolos

% ---
% RESUMOS
% ---

% Resumo, em português, com no máximo 1500 caracteres (com espaço)
% e seis palavras-chaves para o assunto abordado.
\textoresumo[brazil]{
    Este trabalho é um breve modelo  para a escrita de monografias de qualificação, dissertações e teses utilizando o ambiente \LaTeX, de acordo com as normas exigidas pelo Instituto de Ciências Matemáticas e de Computação (ICMC), da Universidade de São Paulo (USP). Para a confecção deste modelo foi utilizado a última versão (1.9.6) do pacote de classes \textit{abnTeX2} que segue as normas da Associação Brasileira de Normas Técnicas. A elaboração de uma monografia, dissertação ou tese pode ser feita sobrescrevendo o conteúdo deste modelo.
    }{Modelo, Monografia de qualificação, Dissertação, Tese, Latex}


% Abstract, em inglês, com no máximo 1500 caracteres (com espaço)
% e seis keywords para o assunto abordado.
\textoresumo[english]{
    This paper is a brief model for writing qualification monographs, dissertations and thesis using \LaTeX environment, in accordance with the standards required by the Institute of Mathematics and Computer Sciences (ICMC), University of São Paulo (USP). For making this model, the latest version (1.9.6) \textit{abnTeX2} classes package was used. This package follow the rules of the Brazilian Association of Technical Standards. A drafting a monograph, dissertation or thesis can be done by overwriting the contents of this model.
    }{Template, Qualification monograph, Dissertation, Thesis, Latex}


% ----
% Início do documento
% ----
\begin{document}
% ----------------------------------------------------------
% ELEMENTOS TEXTUAIS
% ----------------------------------------------------------
\textual

% O trabalho deve conter:
%(a) Apresentação do problema central do trabalho de formatura;
%(b) Formulação no formato de um problema de engenharia;
%(c) Definição de escopo;
%(d) Revisão de literatura;
%(e) Desenvolvimento metodológico (inclusive com a parte experimental quando pertinente);

\chapter{Introdução}
\label{chapter:introducao}
% Comando simples para exibir comandos Latex no texto
\newcommand{\comando}[1]{\textbf{$\backslash$#1}}

Introducao

% Corpo do trabalho (demais capítulos)
\chapter{Revisao de Literatura}
\label{chapter:revisao-literatura}
Revisao da literatura

\chapter{Devenvolvimento metodológico}
\label{chapter:desenvolvimento-metodologico}
Desenvolvimento metodologico

\chapter{Conclusões}
\label{chapter:conclusoes}
Coloque suas conclusoes aqui

% ---
% Finaliza a parte no bookmark do PDF, para que se inicie o bookmark na raiz
% ---
\bookmarksetup{startatroot}%
% ---

% ----------------------------------------------------------
% ELEMENTOS PÓS-TEXTUAIS
% ----------------------------------------------------------
\postextual

% ----------------------------------------------------------
% Anexos - Quando couber
% ----------------------------------------------------------

% ---
% Inicia os anexos
% ---
\begin{anexosenv}

    \chapter{Páginas interessantes na Internet}
    \label{chapter:paginas-interessantes}
    \begin{description}
 \item[\url{http://www.tex-br.org}] Página em português com diversos tutoriais e referências interessantes sobre \LaTeX;
 \item[\url{http://en.wikibooks.org/wiki/LaTeX}] Livro em formato \textit{wiki} gratuito sobre \LaTeX;
 \item[\url{http://tobi.oetiker.ch/lshort/lshort.pdf}] Ótimo tutorial sobre \LaTeX (possui versão em português \url{http://alfarrabio.di.uminho.pt/~albie/lshort/ptlshort.pdf}, mas a versão em inglês é a mais atual);
 \item[\url{http://code.google.com/p/abntex2/}] Página do abnTeX2, grupo que desenvolve os pacotes e classes em \LaTeX para as normas da ABNT, nos quais a classe \textit{icmc} foi baseada;
\item[\url{http://www.more.ufsc.br}] Página do Mecanismo On-line para Referências  (MORE) desenvolvido pela UFSC;
\item[\url{http://detexify.kirelabs.org/classify.html}] Página para recuperar o código de símbolos em \LaTeX a partir do desenho fornecido pelo usuário.
 \end{description}

\end{anexosenv}
% ---

% ----------------------------------------------------------
% Referências bibliográficas
% ----------------------------------------------------------
\bibliography{references}

% ----------------------------------------------------------
% Bibliografia Consultada - OPCIONAL
% ----------------------------------------------------------

% ----------------------------------------------------------
% Apêndices - OPCIONAL
% ----------------------------------------------------------

% ---
% Inicia os apêndices
% ---
\begin{apendicesenv}

    \chapter{Documento básico usando a classe \textit{icmc}}
    \label{chapter:documento-basico}
    \definecolor{gray}{rgb}{0.4,0.4,0.4}
\definecolor{darkblue}{rgb}{0.0,0.0,0.6}
\definecolor{cyan}{rgb}{0.0,0.6,0.6}
\definecolor{maroon}{rgb}{0.5,0,0}
\definecolor{darkgreen}{rgb}{0,0.5,0}


\lstdefinelanguage{myLatex}
{
    keywords={\titulo},
    alsoletter={-},
    sensitive=false,
    morecomment=[l]{\%},
    morecomment=[s]{/*}{*/},
    morestring=[b]",
    morestring=[b]',
    keywordstyle=\bfseries\color{blue},
    commentstyle=\itshape\color{darkgreen},
    morekeywords={documentclass, titulo, autor, data, orientador, coorientador, curso, textoresumo, incluifichacatalografica, textodedicatoria*, textoagradecimentos*, textoepigrafe*, incluilistadefiguras, incluilistadetabelas, incluilistadequadros, incluilistadealgoritmos, incluilistadecodigos, incluilistadesiglas, incluilistadesimbolos, textual, chapter, postextual, begin, bibliography, end}, 
alsoletter={*, \{, \}, \[, \]},
 morekeywords=[2]{\{, \}, \[, \]},
 keywordstyle=[2]\bfseries\color{blue},
 moredelim=[s][\color{maroon}]{\{}{\}},
    moredelim=[s][\itshape\color{maroon}]{\[}{\]},
}

%\lstdefinelanguage{TeX}
%{
%moredelim=*[s][\color{maroon}]{\{}{\}}
%otherkeywords={\{, \}, \[, \], \\}
%  morestring=[b]",
%  moredelim=[s][\bfseries\color{maroon}]{<}{\ },
%  moredelim=[s][\bfseries\color{maroon}]{</}{>},
%  moredelim=[l][\bfseries\color{maroon}]{/>},
%  moredelim=[l][\bfseries\color{maroon}]{>},
%  commentstyle=\color{darkgreen},
%  stringstyle=\color{blue},
%  identifierstyle=\color{red},
%  keywordstyle=\bfseries\color{maroon}
%moredelim=[l][\bfseries\color{maroon}]{>},
%commentstyle=\color{darkgreen},
%  stringstyle=\color{blue},
%  identifierstyle=\color{red}, moredelim=[l][\bfseries\color{maroon}]{\{},
%  keywordstyle=\bfseries\color{maroon}
%}

%\lstset{language={[LaTeX]TeX},
%texcsstyle=*\bfseries\color{blue},
%keywordstyle=\bfseries\color{blue},
%commentstyle=\color{darkgreen},
%morecomment=[s][\color{red}]{\{}{\}},
%otherkeywords={$, \{, \}, \[, \]}
%}

%\begin{codigo}[caption={Exemplo de um documento básico}, label={codigo:documento-basico}, language={[LaTeX]TeX},  breaklines=true,morekeywords={titulo, autor, data, orientador, coorientador, curso, textoresumo, incluifichacatalografica, textodedicatoria*, textoagradecimentos*, textoepigrafe*, incluilistadefiguras, incluilistadetabelas, incluilistadequadros, incluilistadealgoritmos, incluilistadecodigos, incluilistadesiglas, incluilistadesimbolos, {\backslash}textual, chapter, postextual}, alsoletter={{\backslash},*},morecomment=[s][\color{red}]{\{}{\}}]
\begin{codigo}[caption={Exemplo de um documento básico}, label={codigo:documento-basico}, language={myLatex},  breaklines=true]
% Documento utilizando a classe icmc
% Opções: 
%   Qualificação          = qualificacao 
%   Curso                 = doutorado/mestrado
%   Situação do trabalho  = pre-defesa/pos-defesa (exceto para qualificação)
%   Versão para impressão = impressao
\ documentclass[doutorado, pos-defesa]{packages/icmc}

% Título do trabalho em Português
\tituloPT{Título da Monografia}

% Título do trabalho em Inglês
\tituloEN{Título da Monografia}

% Nome do autor
\autor[Abreviação]{Nome completo do autor}

% Gênero do autor (M ou F)
\genero{M}

% Data do depósito
\data{18}{12}{2012}

% Nome do Orientador
\orientador[Orientador]{Titulação do orientador}{Nome completo do Orientador}

% Nome do Coorientador (caso não exista basta remover)
\coorientador[Coorientador]{Titulação do coorientador}{Nome completo do Coorientador}
% Se coorientadora troque Coorientador: por Coorientadora dentro do colchetes

% Sigla do programa de Pós-graduação (CCMC, MAT, PIPGES, PROFMAT, MECAI)
\curso{CCMC}
% O valor entre colchetes é opcional para este programa

% Idioma principal do texto (EN ou PT)
\idioma{PT}

% Resumo
\textoresumo[Idioma]{
Texto do resumo do trabalho.
}{Lista de palavras-chave separada por virgulas}

% ----------------------------------------------------------
% ELEMENTOS PRÉ-TEXTUAIS
% ----------------------------------------------------------

% Inserir a ficha catalográfica
\incluifichacatalografica{tex/ficha-catalografica.pdf}

% Incluí o texto da Dedicatória
\textodedicatoria*{tex/pre-textual/dedicatoria}

% Incluí o texto dos Agradecimentos
\textoagradecimentos*{tex/pre-textual/agradecimentos}

% Incluí o texto da Epígrafe
\textoepigrafe*{tex/pre-textual/epigrafe}

% Inclui a lista de figuras
\incluilistadefiguras

% Inclui a lista de tabelas
\incluilistadetabelas

% Inclui a lista de quadros
\incluilistadequadros

% Inclui a lista de algoritmos
\incluilistadealgoritmos

% Inclui a lista de códigos
\incluilistadecodigos

% Inclui a lista de siglas e abreviaturas
\incluilistadesiglas

% Inclui a lista de símbolos
\incluilistadesimbolos

% Início do documento
\begin{document}

% ----------------------------------------------------------
% ELEMENTOS TEXTUAIS
% ----------------------------------------------------------
\textual

\chapter{Introdução}

Capítulo de Introdução

\chapter{Desenvolvimento}

Capítulo de Desenvolvimento

\chapter{Conclusão}

Capítulo de conclusão

% ----------------------------------------------------------
% ELEMENTOS PÓS-TEXTUAIS
% ----------------------------------------------------------
\postextual

% Nome do arquivo com as referências bibliográficas
\bibliography{referencias}

\end{document}

\end{codigo}

    \chapter{Configuração do programa JabRef}
    \label{chapter:configuracao-jabref}
    \lstdefinelanguage{XML}
{
  morestring=[b]",
  moredelim=[s][\bfseries\color{maroon}]{<}{\ },
  moredelim=[s][\bfseries\color{maroon}]{</}{>},
  moredelim=[l][\bfseries\color{maroon}]{/>},
  moredelim=[l][\bfseries\color{maroon}]{>},
  morecomment=[s]{<?}{?>},
  morecomment=[s]{<!--}{-->},
  commentstyle=\color{darkgreen},
  stringstyle=\color{blue},
  identifierstyle=\color{red}
}


\begin{codigo}[caption={Código de configuração do programa JabRef em XML}, label={codigo:config-jabref}, language=XML, breaklines=true]
<?xml version="1.0" encoding="UTF-8" standalone="no"?>
<!DOCTYPE preferences SYSTEM "http://java.sun.com/dtd/preferences.dtd">
<preferences EXTERNAL_XML_VERSION="1.0">
  <root type="user">
    <map/>
    <node name="net">
      <map/>
      <node name="sf">
        <map/>
        <node name="jabref">
          <map>
            <entry key="KeyPatternRegex" value=""/>
            <entry key="KeyPatternReplacement" value=""/>
            <entry key="abbrAuthorNames" value="true"/>
            <entry key="allowTableEditing" value="false"/>
            <entry key="autoComplete" value="true"/>
            <entry key="autoCompleteFields" value="author;editor;title;journal;publisher;keywords;crossref"/>
            <entry key="autoDoubleBraces" value="true"/>
            <entry key="autoOpenForm" value="true"/>
            <entry key="autoResizeMode" value="4"/>
            <entry key="autoSave" value="true"/>
            <entry key="autoSaveInterval" value="5"/>
            <entry key="autolinkExactKeyOnly" value="true"/>
            <entry key="avoidOverwritingKey" value="false"/>
            <entry key="backup" value="false"/>
            <entry key="caseSensitiveSearch" value="false"/>
            <entry key="citeseerColumn" value="false"/>
            <entry key="confirmDelete" value="true"/>
            <entry key="ctrlClick" value="false"/>
            <entry key="customTypeName_0" value="Article"/>
            <entry key="customTypeName_1" value="Book"/>
            <entry key="customTypeName_10" value="Misc"/>
            <entry key="customTypeName_11" value="Monography"/>
            <entry key="customTypeName_12" value="Patent"/>
            <entry key="customTypeName_13" value="Periodical"/>
            <entry key="customTypeName_14" value="Phdthesis"/>
            <entry key="customTypeName_15" value="Proceedings"/>
            <entry key="customTypeName_16" value="Standard"/>
            <entry key="customTypeName_17" value="Techreport"/>
            <entry key="customTypeName_2" value="Booklet"/>
            <entry key="customTypeName_3" value="Conference"/>
            <entry key="customTypeName_4" value="Electronic"/>
            <entry key="customTypeName_5" value="Inbook"/>
            <entry key="customTypeName_6" value="Incollection"/>
            <entry key="customTypeName_7" value="Inproceedings"/>
            <entry key="customTypeName_8" value="Manual"/>
            <entry key="customTypeName_9" value="Mastersthesis"/>
            <entry key="customTypeOpt_0" value="month;part;section;url;urlaccessdate;note"/>
            <entry key="customTypeOpt_1" value="subtitle;edition;pages;number;series;isbn;volume;org-short;url;urlaccessdate;note"/>
            <entry key="customTypeOpt_10" value="howpublished;month;year;publisher;subtitle;pages;pagename;address;series;number;editortype;url;urlaccessdate;note"/>
            <entry key="customTypeOpt_11" value="pages;pagename;url;urlaccessdate;note"/>
            <entry key="customTypeOpt_12" value="author;title;language;assignee;address;type;number;day;dayfiled;month;monthfiled;url;note"/>
            <entry key="customTypeOpt_13" value="editor;language;series;volume;number;organization;month;url;org-short;note"/>
            <entry key="customTypeOpt_14" value="pages;pagename;url;urlaccessdate;note"/>
            <entry key="customTypeOpt_15" value="editor;volume;number;series;address;publisher;month;organization;org-short;note"/>
            <entry key="customTypeOpt_16" value="author;language;howpublished;type;number;revision;address;month;year;url;org-short;note"/>
            <entry key="customTypeOpt_17" value="pages;pagename;org-short;url;urlaccessdate;number;month;note"/>
            <entry key="customTypeOpt_2" value="subtitle;edition;pages;number;volume;org-short;url;urlaccessdate;note"/>
            <entry key="customTypeOpt_3" value="editor;volume;number;series;pages;address;month;organization;publisher;org-short;note"/>
            <entry key="customTypeOpt_4" value="month;year;org-short;note"/>
            <entry key="customTypeOpt_5" value="booksubtitle;edition;number;series;isbn;volume;org-short;editortype;url;urlaccessdate;note"/>
            <entry key="customTypeOpt_6" value="booksubtitle;edition;number;series;isbn;volume;org-short;editortype;url;urlaccessdate;note"/>
            <entry key="customTypeOpt_7" value="pages;month;publisher;booktitle;conference-location;conference-year;url;urlaccessdate;note"/>
            <entry key="customTypeOpt_8" value="subtitle;author;organization;org-short;address;edition;month;year;pages;series;url;urlaccessdate;note"/>
            <entry key="customTypeOpt_9" value="pages;pagename;url;urlaccessdate;note"/>
            <entry key="customTypeReq_0" value="author;title;journal;year;volume;number;pages"/>
            <entry key="customTypeReq_1" value="title;author/editor/organization;publisher;year;address"/>
            <entry key="customTypeReq_10" value=";author/organization/editor/title"/>
            <entry key="customTypeReq_11" value="author;title;type;school;year;address"/>
            <entry key="customTypeReq_12" value="nationality;number;year;yearfiled"/>
            <entry key="customTypeReq_13" value="title;year"/>
            <entry key="customTypeReq_14" value="author;title;school;year;address"/>
            <entry key="customTypeReq_15" value="title;year"/>
            <entry key="customTypeReq_16" value="title;organization/institution"/>
            <entry key="customTypeReq_17" value="author;title;organization/school;year;address"/>
            <entry key="customTypeReq_2" value="title;author/editor/organization;year"/>
            <entry key="customTypeReq_3" value="author;title;booktitle;year"/>
            <entry key="customTypeReq_4" value="url;urlaccessdate;author/organization/title"/>
            <entry key="customTypeReq_5" value="author;title;editor/organization;booktitle;chapter/pages;publisher;address;year"/>
            <entry key="customTypeReq_6" value="author;title;booktitle;editor/organization;chapter/pages;publisher;address;year"/>
            <entry key="customTypeReq_7" value="author;title;organization;conference-number;year;address"/>
            <entry key="customTypeReq_8" value="title"/>
            <entry key="customTypeReq_9" value="author;title;school;year;address"/>
            <entry key="defaultEncoding" value="ISO8859_15"/>
            <entry key="defaultLabelPattern" value="[auth]:[year]"/>
            <entry key="defaultOwner" value=""/>
            <entry key="defaultShowSource" value="false"/>
            <entry key="dialogWarningForDuplicateKey" value="true"/>
            <entry key="dialogWarningForEmptyKey" value="true"/>
            <entry key="disableOnMultipleSelection" value="false"/>
            <entry key="doNotResolveStringsFor" value="url"/>
            <entry key="enableSourceEditing" value="true"/>
            <entry key="enforceLegalBibtexKey" value="true"/>
            <entry key="exportInOriginalOrder" value="false"/>
            <entry key="exportInStandardOrder" value="true"/>
            <entry key="exportWorkingDirectory" value="/home/marcos/tmp"/>
            <entry key="fileColumn" value="true"/>
            <entry key="fileDirectory" value=""/>
            <entry key="filechooserDisableRename" value="true"/>
            <entry key="floatMarkedEntries" value="true"/>
            <entry key="floatSearch" value="true"/>
            <entry key="fontFamily" value="SansSerif"/>
            <entry key="fontSize" value="12"/>
            <entry key="fontStyle" value="0"/>
            <entry key="generateKeysAfterInspection" value="true"/>
            <entry key="generateKeysBeforeSaving" value="false"/>
            <entry key="gridColor" value="210:210:210"/>
            <entry key="groupAutoHide" value="true"/>
            <entry key="groupAutoShow" value="true"/>
            <entry key="groupExpandTree" value="true"/>
            <entry key="groupKeywordSeparator" value=", "/>
            <entry key="groupShowDynamic" value="true"/>
            <entry key="groupShowIcons" value="true"/>
            <entry key="groupsDefaultField" value="keywords"/>
            <entry key="incompleteEntryBackground" value="250:175:175"/>
            <entry key="incrementS" value="false"/>
            <entry key="lastEdited" value="/home/marcos/Documentos/IFMG/Acadêmico/Aulas/Latex/ifmgbitex/referencias.bib"/>
            <entry key="lastUsedExport" value="html"/>
            <entry key="lookAndFeel" value="com.jgoodies.plaf.plastic.Plastic3DLookAndFeel"/>
            <entry key="markImportedEntries" value="true"/>
            <entry key="markedEntryBackground" value="255:255:180"/>
            <entry key="memoryStickMode" value="false"/>
            <entry key="namesAsIs" value="false"/>
            <entry key="namesFf" value="false"/>
            <entry key="namesLastOnly" value="false"/>
            <entry key="namesNatbib" value="true"/>
            <entry key="openLastEdited" value="true"/>
            <entry key="overrideDefaultFonts" value="false"/>
            <entry key="overwriteOwner" value="false"/>
            <entry key="overwriteTimeStamp" value="false"/>
            <entry key="pdfColumn" value="false"/>
            <entry key="pdfDirectory" value=""/>
            <entry key="posX" value="0"/>
            <entry key="posY" value="0"/>
            <entry key="preview0" value="&lt;font face=&quot;arial&quot;&gt;&lt;b&gt;&lt;i&gt;\bibtextype&lt;/i&gt;&lt;a name=&quot;\bibtexkey&quot;&gt;\begin{bibtexkey} (\bibtexkey)&lt;/a&gt;\end{bibtexkey}&lt;/b&gt;&lt;br&gt;__NEWLINE__\begin{author} \format[HTMLChars,AuthorAbbreviator,AuthorAndsReplacer]{\author}&lt;BR&gt;\end{author}__NEWLINE__\begin{editor} \format[HTMLChars,AuthorAbbreviator,AuthorAndsReplacer]{\editor} &lt;i&gt;(\format[IfPlural(Eds.,Ed.)]{\editor})&lt;/i&gt;&lt;BR&gt;\end{editor}__NEWLINE__\begin{title} \format[HTMLChars]{\title} \end{title}&lt;BR&gt;__NEWLINE__\begin{chapter} \format[HTMLChars]{\chapter}&lt;BR&gt;\end{chapter}__NEWLINE__\begin{journal} &lt;em&gt;\format[HTMLChars]{\journal}, &lt;/em&gt;\end{journal}__NEWLINE__\begin{booktitle} &lt;em&gt;\format[HTMLChars]{\booktitle}, &lt;/em&gt;\end{booktitle}__NEWLINE__\begin{school} &lt;em&gt;\format[HTMLChars]{\school}, &lt;/em&gt;\end{school}__NEWLINE__\begin{institution} &lt;em&gt;\format[HTMLChars]{\institution}, &lt;/em&gt;\end{institution}__NEWLINE__\begin{publisher} &lt;em&gt;\format[HTMLChars]{\publisher}, &lt;/em&gt;\end{publisher}__NEWLINE__\begin{year}&lt;b&gt;\year&lt;/b&gt;\end{year}\begin{volume}&lt;i&gt;, \volume&lt;/i&gt;\end{volume}\begin{pages}, \format[FormatPagesForHTML]{\pages} \end{pages}__NEWLINE__\begin{abstract}&lt;BR&gt;&lt;BR&gt;&lt;b&gt;Abstract: &lt;/b&gt; \format[HTMLChars]{\abstract} \end{abstract}__NEWLINE__\begin{review}&lt;BR&gt;&lt;BR&gt;&lt;b&gt;Review: &lt;/b&gt; \format[HTMLChars]{\review} \end{review}&lt;/dd&gt;__NEWLINE__&lt;p&gt;&lt;/p&gt;&lt;/font&gt;"/>
            <entry key="preview1" value="&lt;font face=&quot;arial&quot;&gt;&lt;b&gt;&lt;i&gt;\bibtextype&lt;/i&gt;&lt;a name=&quot;\bibtexkey&quot;&gt;\begin{bibtexkey} (\bibtexkey)&lt;/a&gt;\end{bibtexkey}&lt;/b&gt;&lt;br&gt;__NEWLINE__\begin{author} \format[HTMLChars,AuthorAbbreviator,AuthorAndsReplacer]{\author}&lt;BR&gt;\end{author}__NEWLINE__\begin{editor} \format[HTMLChars,AuthorAbbreviator,AuthorAndsReplacer]{\editor} &lt;i&gt;(\format[IfPlural(Eds.,Ed.)]{\editor})&lt;/i&gt;&lt;BR&gt;\end{editor}__NEWLINE__\begin{title} \format[HTMLChars]{\title} \end{title}&lt;BR&gt;__NEWLINE__\begin{chapter} \format[HTMLChars]{\chapter}&lt;BR&gt;\end{chapter}__NEWLINE__\begin{journal} &lt;em&gt;\format[HTMLChars]{\journal}, &lt;/em&gt;\end{journal}__NEWLINE__\begin{booktitle} &lt;em&gt;\format[HTMLChars]{\booktitle}, &lt;/em&gt;\end{booktitle}__NEWLINE__\begin{school} &lt;em&gt;\format[HTMLChars]{\school}, &lt;/em&gt;\end{school}__NEWLINE__\begin{institution} &lt;em&gt;\format[HTMLChars]{\institution}, &lt;/em&gt;\end{institution}__NEWLINE__\begin{publisher} &lt;em&gt;\format[HTMLChars]{\publisher}, &lt;/em&gt;\end{publisher}__NEWLINE__\begin{year}&lt;b&gt;\year&lt;/b&gt;\end{year}\begin{volume}&lt;i&gt;, \volume&lt;/i&gt;\end{volume}\begin{pages}, \format[FormatPagesForHTML]{\pages} \end{pages}&lt;/dd&gt;__NEWLINE__&lt;p&gt;&lt;/p&gt;&lt;/font&gt;"/>
            <entry key="priDescending" value="false"/>
            <entry key="priSort" value="entrytype"/>
            <entry key="promptBeforeUsingAutosave" value="true"/>
            <entry key="psDirectory" value=""/>
            <entry key="pushToApplication" value="Insert selected citations into LyX/Kile"/>
            <entry key="recentFiles" value="/home/marcos/Documentos/IFMG/Acadêmico/Aulas/Algoritmos/Algoritmos_exercicios_01/referencias.bib;/home/marcos/Documentos/IFMG/TCC e Projetos/ERP Comparativo/referencias.bib"/>
            <entry key="regExpSearch" value="true"/>
            <entry key="rememberWindowLocation" value="true"/>
            <entry key="resolveStringsAllFields" value="false"/>
            <entry key="runAutomaticFileSearch" value="false"/>
            <entry key="saveInOriginalOrder" value="false"/>
            <entry key="saveInStandardOrder" value="true"/>
            <entry key="searchAll" value="false"/>
            <entry key="searchAllBases" value="false"/>
            <entry key="searchGen" value="true"/>
            <entry key="searchOpt" value="true"/>
            <entry key="searchPanelVisible" value="false"/>
            <entry key="searchReq" value="true"/>
            <entry key="secDescending" value="false"/>
            <entry key="secSort" value=""/>
            <entry key="selectS" value="false"/>
            <entry key="showSearchInDialog" value="false"/>
            <entry key="showSource" value="true"/>
            <entry key="sizeX" value="1280"/>
            <entry key="sizeY" value="800"/>
            <entry key="stringsPosX" value="340"/>
            <entry key="stringsPosY" value="200"/>
            <entry key="stringsSizeX" value="600"/>
            <entry key="stringsSizeY" value="400"/>
            <entry key="tableBackground" value="255:255:255"/>
            <entry key="tableColorCodesOn" value="true"/>
            <entry key="tableOptFieldBackground" value="230:255:230"/>
            <entry key="tableReqFieldBackground" value="230:235:255"/>
            <entry key="tableText" value="0:0:0"/>
            <entry key="terDescending" value="false"/>
            <entry key="terSort" value=""/>
            <entry key="timeStampField" value="timestamp"/>
            <entry key="timeStampFormat" value="dd/MM/yyyy"/>
            <entry key="unmarkAllEntriesBeforeImporting" value="true"/>
            <entry key="urlColumn" value="true"/>
            <entry key="useDefaultLookAndFeel" value="true"/>
            <entry key="useIEEEAbrv" value="true"/>
            <entry key="useImportInspectionDialog" value="true"/>
            <entry key="useImportInspectionDialogForSingle" value="true"/>
            <entry key="useNativeFileDialogOnMac" value="false"/>
            <entry key="useOwner" value="false"/>
            <entry key="useRegExpSearch" value="false"/>
            <entry key="useRemoteServer" value="false"/>
            <entry key="useTimeStamp" value="true"/>
            <entry key="useXmpPrivacyFilter" value="false"/>
            <entry key="warnAboutDuplicatesInInspection" value="true"/>
            <entry key="warnBeforeOverwritingKey" value="true"/>
            <entry key="windowMaximised" value="false"/>
            <entry key="workingDirectory" value="/home/marcos/Documentos/IFMG/Acadêmico/Aulas/Algoritmos/Algoritmos_exercicios_01"/>
          </map>
          <node name="labelPattern">
            <map/>
          </node>
        </node>
      </node>
    </node>
  </root>
</preferences>

\end{codigo}

\end{apendicesenv}
% ---

% ---------------------------------------------------------------------
% GLOSSÁRIO -OPCIONAL
% ---------------------------------------------------------------------

% Arquivo que contém as definições que vão aparecer no glossário
\newword{WYSIWYG}{``What You See Is What You Get''  ou ``O que você vê é o que você obtém''.  Recurso tem por objetivo permitir que um documento, enquanto manipulado na tela, tenha a mesma aparência de sua utilização, usualmente sendo considerada final. Isso facilita para o desenvolvedor que pode trabalhar visualizando a aparência do documento sem precisar salvar em vários momentos e abrir em um \textit{software} separado de visualização}
\newword{Framework}{é uma abstração que une códigos comuns entre vários projetos de \textit{software} provendo uma funcionalidade genérica. \textit{Frameworks} são projetados com a intenção de facilitar o desenvolvimento de \textit{software}, habilitando designers e programadores a gastarem mais tempo determinando as exigências do \textit{software} do que com detalhes de baixo nível do sistema}

\newword{Template}{é um documento sem conteúdo, com apenas a apresentação visual (apenas cabeçalhos por exemplo) e instruções sobre onde e qual tipo de conteúdo deve entrar a cada parcela da apresentação}

\newword{Padrões de projeto}{ou \textit{Design Pattern}, descreve uma solução geral reutilizável para um problema recorrente no desenvolvimento de sistemas de \textit{software} orientados a objetos. Não é um código final, é uma descrição ou modelo de como resolver o problema do qual trata, que pode ser usada em muitas situações diferentes}

\newword{Web}{Sinônimo mais conhecido de \textit{World Wide Web} (WWW). É a interface gráfica da Internet que torna os serviços disponíveis totalmente transparentes para o usuário e ainda possibilita a manipulação multimídia da informação}

% Comando para incluir todas as definições do arquivo glossario.tex
\glsaddall
% Impressão do glossário
\printglossaries


\end{document}